\graphicspath{{introduction/fig/}}


\chapter{Introduction} 
\label{chap:introduction}
\section{Background}

In the current climate of technology, systems and devices centred around the Internet of Things concept is on the rise.
There are more and more consumer and commercial products that are connected to the internet in some way or another, and most of these devices are controlled via an accompanying website or mobile phone application.
This is a very effective and comfortable way of always being on control of your devices and environment, and almost everything in a consumer home can now be connected to the internet, from your fridge to your lightbulbs to your water-meter.

One aspect of a consumer's home that hasn't received an all-inclusive IoT upgrade is houseplants.
Houseplants and desktop plants are arguably the easier plants to grow, but there is still a need to monitor them to ensure they can grow healthy. 
Most houseplants do require a lot of water and can sometimes be forgotten about and are difficult to get healthy again.
There are numerous methods to provide solutions, but none have been comprehensively explored and put to market.

\section{Problem Statement}

The aim is to design a smart desktop plant watering system to monitor a couple of plants' health and report it to a website or app. 
The system must also document the growth of the plants via a camera 

\section{Project Objective}
  
The project sets out to fulfil the following objectives. The system must:

\begin{itemize}
    \item measure the soil moisture content of the plant soil, with the possibility of viewing the plant health using a camera system. 
    \item be able to activate a watering system that is able to water each plant individually.
    \item display this information via a website or app, allowing the user to activate the watering for each individual plant from the website.
    \item store camera photos of the plants in order to create a time-lapse video of the plants' growth.   
\end{itemize}

\section{Summary}



\section{Scope}

The project sets out to solve the problem in the most comprehensive way possible while also allowing for a user-friendly experience. 
There are some solutions that, although possible, would not make sense to implement.

One such a solution is to not only monitor the soil moisture content, but also the \pH levels of the soil. 
This allows for comprehensive data of the plants' health and soil mixture, but it is very difficult to implement in a digital system in a way that makes sense and would be useful to the user.
A simpler approach to this would be to use \pH and nutrient sticks purchased from a local plant nursery.

The project also does not aim to adhere to a specific cost cap, but materials and components are chosen in such a way that they try to find the middle-ground between value and cost. The components are also limited to a pool that is available at time of writing. Prices are subject to change as water pumps improve and as microcontrollers and camera modules become more affordable.

\section{Layout of this report}

